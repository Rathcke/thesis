\section{Related work}\label{survey}

\subsection{The general case}
\subsubsection{Approximate distance oracles}
Hermelin et al. \cite{hermelin2011distance} proposed several results for approximate
distance oracles. In the first result, they give a vertex-labeled distance oracle of
expected size $O(kn^{1+1/k})$ with a stretch of $(4k-5)$ and query time $O(k)$. This can
be constructed in $O(kmn^{1/k})$ time. The data structure is an adaption of Thorup and
Zwick's distance oracle \cite{thorup2005approximate}. \\
In another result, they give a distance oracle which depend on both $n$ and $\ell$. This has
expected size $O(knl^{1/k})$ with query time $O(k)$, but with an exponential stretch of
$(2^k-1)$. This is constructed in $O(kmn^{k/(2k-1)})$. \\
In their third result, they give an oracle capable of handling label changes. This has
expected space $O(kn^{1+1/k})$, stretch $(2\cdot3^{k-1}+1)$, query time $O(k)$ and
handles label changes in $O(kn^{1/k}\lg \lg n)$. \\
\\
Chechik \cite{chechik2012improved} improved the latter two results. Namely, she managed to
reduce the stretch from exponential down to polynomial. She showed two theorems that can
be stated as follows:
\begin{thm}\label{chech1}
  There is a vertex-labeled distance oracle
of size $O(kn\ell^{1/k})$, stretch $(4k-5)$  and query time $O(k)$, which can be
constructed in $O(m\cdot \min\{n^{k/(2k-1)}, \ell\})$.
\end{thm}
\begin{thm}\label{chech2}
  There is a vertex-labeled and dynamic vertex-labeled distance oracle capable of handling label changes in
$\tilde{O}(n^{1/k})$ time with stretch $(4k-5)$ and space $\tilde{O}(n^{1+1/k})$.
\end{thm}
For Theorem \ref{chech1}, the constructed data structure is similar to that introduced
by Hermelin et al. \cite{hermelin2011distance} (which in turn is an adaption of the data
structure given by Thorup and Zwick in \cite{thorup2005approximate}). \\
TODO: fig of results (approximate)

\subsection{The planar case}\label{cohenplanar}
Cohen-Addad et al. \cite{cohen2017fast} introduced a (non vertex-labeled) distance
oracle in planar digraphs that required subquadratic space and could answer exact distance queries in
$O(\lg n)$ time. This was the first result to achieve polylogarithmic query time, as previous
results required query time polynomial in $n$ or could only return approximate distances.
This result was later improved in \cite{gawrychowski2017better} from using $O(n^{5/3})$
space to $O(n^{3/2})$ space. They also improved the tradeoff for any space $S\in [n,n^2]$.
That is, for $S\in [n, n^{3/2}]$, they give an oracle with query time
$\tilde{O}(n^{3/2}/S)$. For $S\in [n^{3/2}, n^2]$, they give an oracle with query time
$O(\lg n)$. \\
\\
TODO:
\begin{itemize}
  \item Recursive decomposition using Jordan curves.
  \item Voronoi diagrams
  \item Distances stored
  \item Point location approach
\end{itemize}
We give two exact vertex-labeled distance oracles for planar graphs in \ref{exactPlanar}. In
\ref{techniques} we introduce techniques commonly used in directed planar graphs.

\begin{table}[H]
  \footnotesize
  \centering
  \begin{tabular}{c | c | c | c | c}
    Reference & Dir. & Preprocessing & Space & Query \\
    \hline\hline
    This & D & TODO & $O(n\ell^{2/3})$ & $O(\ell^{1/3})$ \\
    \hline
    This & D & TODO & $O(n^{3/2})$ & $O(\text{polylog}(n))$ \\
    \hline
    Mozes et al. \cite{mozes2015efficient} & U & $O(\varepsilon^{-2}n\lg^3 n)$ & $O(\varepsilon^{-1}n\lg n)$ &
    $O(\lg \lg n + \varepsilon^{-1})$ \\
    \hline
    Mozes et al. \cite{mozes2015efficient} & D &
    $O(\varepsilon^{-2}n\lg^3 n\lg (nN))$ & $O(\varepsilon^{-1}n\lg n \lg(nN))$ & $O(\lg
    \lg n \lg \lg (nN) + \varepsilon^{-1})$ \\
    \hline
    Li et al. \cite{li20131+} & U & $O(\varepsilon^{-1}n\lg^2 n)$ & $O(\varepsilon^{-1}n\lg n)$ & $O(\varepsilon^{-1}\lg
    n \lg \Delta)$
  \end{tabular}
  \caption{Results for vertex-labeled distance oracles in planar graph. All results are
  for connected graphs. All results but those introduced in this thesis are
$(1+\varepsilon)$ stretch results. The letters 'U' and 'D' is for undirected and directed
graphs respectively. The 'N' is the maximum arc length and $\Delta$ is the hop-diameter
of the graph.}
  \label{planarresults}
\end{table}
