\section{Related work}\label{survey}
Distance oracles were introduced by Thorup and Zwick \cite{thorup2005approximate} as a
faster alternative to the SSSP and APSP algorithms. They proposed an approximate distance
oracle, which was later the inspiration for Hermelin et al. when they introduced distance
oracles for vertex-labeled graphs. \\
Hermelin et al. \cite{hermelin2011distance} proposed several results for approximate
distance oracles. In the first result, they give a vertex-labeled distance oracle of
expected size $O(kn^{1+1/k})$ with a stretch of $(4k-5)$ and query time $O(k)$. This can
be constructed in $O(kmn^{1/k})$ time. The data structure is an adaption of Thorup and
Zwick's distance oracle \cite{thorup2005approximate}. \\
In another result, they give a distance oracle which depend on both $n$ and $\ell$. This has
expected size $O(knl^{1/k})$ with query time $O(k)$, but with an exponential stretch of
$(2^k-1)$. This is constructed in $O(kmn^{k/(2k-1)})$. \\
In their third result, they give an oracle capable of handling label changes. This has
expected space $O(kn^{1+1/k})$, stretch $(2\cdot3^{k-1}+1)$, query time $O(k)$ and
handles label changes in $O(kn^{1/k}\lg \lg n)$. \\
\\
Chechik \cite{chechik2012improved} improved the latter two results. Namely, she managed to
reduce the stretch from exponential down to polynomial. She showed two theorems that can
be stated as follows:
\begin{thm}\label{chech1}
  There is a vertex-labeled distance oracle
of size $O(kn\ell^{1/k})$, stretch $(4k-5)$  and query time $O(k)$, which can be
constructed in $O(m\cdot \min\{n^{k/(2k-1)}, \ell\})$.
\end{thm}
\begin{thm}\label{chech2}
  There is a dynamic vertex-labeled distance oracle capable of handling label changes in
$\tilde{O}(n^{1/k})$ time with stretch $(4k-5)$ and space $\tilde{O}(n^{1+1/k})$. It has
query time $O(k)$ and can be constructed in $O(kmn^{1/k})$.
\end{thm}
\noindent
However, the focus of this thesis is on exact distance oracles in planar graphs and the approaches used to
approximate distances are often not applicable to obtain exact distance oracles.
Furthermore, very little is known about exact vertex-labeled distance oracles -
especially in planar graphs. A summary of distance oracles for vertex-labeled graphs in static planar graphs is given in Table
\ref{vertexresults}. Note that the work in the field is very limited and all known
results approximate distances. Below, we outline results for exact distance oracles for planar graphs that are not labeled.

\begin{table}[h!]
  \centering
  \begin{adjustbox}{width=1\textwidth}
  \small
  \begin{tabular}{c | c | c | c}
    Reference & Preprocessing & Space & Query \\
    \hline\hline
    This & $O(n^2/\ell{1/3})$ & $O(n\ell^{2/3})$ & $O(\ell^{1/3})$ \\
    \hline
    This & $O(n^2)$ & $O(n^{3/2})$ & $O(1)$ \\
    \hline
    This & $O(n^{3/2})$ & $O(n^{3/2})$ & $O(\text{polylog}(n))$ \\
    \hline
    Mozes et al. \cite{mozes2015efficient} &
    $O(\varepsilon^{-2}n\lg^3 n\lg (nN))$ & $O(\varepsilon^{-1}n\lg n \lg(nN))$ & $O(\lg
    \lg n \lg \lg (nN) + \varepsilon^{-1})$ \\
    \hline
    Mozes et al. \cite{mozes2015efficient} & $O(\varepsilon^{-2}n\lg^3 n)$ & $O(\varepsilon^{-1}n\lg n)$ &
    $O(\lg \lg n + \varepsilon^{-1})$ \\
    \hline
    Li et al. \cite{li20131+} & $O(\varepsilon^{-1}n\lg^2 n)$ & $O(\varepsilon^{-1}n\lg n)$ & $O(\varepsilon^{-1}\lg
    n \lg \Delta)$
  \end{tabular}
\end{adjustbox}
  \caption{Results for vertex-labeled distance oracles in planar graph. All results are
  for connected graphs. All results but those introduced in this thesis are
$(1+\varepsilon)$ stretch results. The first three results are for directed graphs and
the last two are for undirected graphs. The 'N' is the maximum arc length and $\Delta$ is the hop-diameter
of the graph.}
  \label{vertexresults}
\end{table}

\subsection{The static planar case}\label{surveyplanar}
Djidjev \cite{djidjev1996efficient} worked towards an algorithm which had a space-query
product beat those of SSSP and APSP problems. He showed that, given a desired space
allocation $S\in [n, n^2]$, the distance oracle has $O(S)$ space and $O(n^2/S)$
query time. These results are for any class of graphs where the separator theorem holds. For planar graphs
with $S\in [n^{4/3}, n^{3/2}]$, he gives a distance oracle with query time $O(n\lg
n/\sqrt{S})$. \\
This trade-off, $\tilde{O}(n/\sqrt{S})$, was later improved to work over the entire range
of $S\in [n,n^2]$ in the subsequent papers by Chen and Xu \cite{chen2000shortest},
Fakcharoenphol and Rao \cite{fakcharoenphol2006planar} and Cabello
\cite{cabello2006many}. Mozes and Sommer \cite{mozes2012exact} gave a distance oracle
with this trade-off for $S\in [n, n^{4/3}]$ completing it for the entire range of $S$. Another result is for
linear space and for any constant $\varepsilon > 0$, where they manage to produce a distance oracle with preprocessing time
$O(n\lg n)$ and a $O(n^{1/2+\varepsilon})$ query time (This result
was simultaneously found by Nussbaum \cite{nussbaum2011improved}). \\
\\
For constant query times, the best known result is by Wulff-Nilsen
\cite{wulff2010algorithms} who gives a slightly subquadratic result of
$O(n^2\text{polyloglog}(n)/\lg n)$ space. \\
\\
Cohen-Addad et al. \cite{cohen2017fast} introduced a distance
oracle in planar digraphs that required subquadratic space, $O(n^{5/3})$, and could answer exact distance queries in
$O(\lg n)$ time. This was the first result to achieve polylogarithmic query time, as previous
results required query time polynomial in $n$ or could only return approximate distances.
They also managed to improve the trade-off for $S\geq n^{3/2}$, where they get query
times of $\tilde{O}(n^{5/2}/S^{3/2})$. \\
\\
This result was later improved in \cite{gawrychowski2017better} from using $O(n^{5/3})$
space to $O(n^{3/2})$ space. They also improved the tradeoff for any space $S\in [n,n^2]$.
That is, for $S\in [n, n^{3/2}]$, they give an oracle with query time
$\tilde{O}(n^{3/2}/S)$. For $S\in [n^{3/2}, n^2]$, they give an oracle with query time
$O(\lg n)$. The approach is a point location structure for additively weighted Voronoi
diagrams. They get a recursive construction using Jordan curves as separator and store a
Voronoi diagram for each node $u$ in the graph. The sites correspond to the boundary
vertices and weights are the distances from $u$ to each site. For a query $Q(u,v)$, they
locate a node $v$ in the Voronoi diagram for $u$, from which they can obtain the
distance. Our result in Section \ref{oracle2} applies the techniques used in the work to
get efficient query times for distance oracles in vertex-labeled graphs. \\
\\
A summary of preprocessing, space and query times for distance oracles in planar graphs is given in Table \ref{planarresults}.

\begin{table}[h!]
  \footnotesize
  \centering
  \begin{tabular}{c | c | c | c}
    Reference & Space & Query & Preprocessing \\
    \hline\hline
    Djidjev \cite{djidjev1996efficient} & $S\in [n^{3/2}, n^2]$ & $O(n^2/S)$ & $O(S)$ \\
    \hline
    Djidjev \cite{djidjev1996efficient} & $S\in [n, n^{3/2}]$ & $O(n^2/S)$ & $O(n\sqrt{S})$ \\
    \hline
    Djidjev \cite{djidjev1996efficient} & $S\in [n^{4/3}, n^{3/2}]$ &
    $\tilde{O}(n/\sqrt{S})$ & $O(n\sqrt{S})$ \\
    \hline
    Chen and Xu \cite{chen2000shortest} & $S\in [n^{4/3}, n^2]$ & $\tilde{O}(n/\sqrt{S})$
    & $O(n\sqrt{S})$ \\
    \hline
    Cabello \cite{cabello2006many} & $S\in [n^{4/3}, n^2]$ & $\tilde{O}(n/\sqrt{S})$ &
    $O(S)$ \\
    \hline
    FR-djikstra \& Klein \cite{klein2005multiple}\cite{fakcharoenphol2006planar}& $S = O(n\lg n)$ &
    $O(\sqrt{n}\lg^2 n)$ & $O(n\lg^2 n)$ \\
    \hline
    Mozes and Sommer \cite{mozes2012exact} & $S\in [n, n^{4/3}]$ &
    $\tilde{O}(n/\sqrt{S})$ & $\tilde{O}(S)$ \\
    \hline
    Mozes and Sommer \cite{mozes2012exact} & $S=O(n)$ & $O(n^{1/2+\varepsilon})$ & $O(n\lg
    n)$ \\
    \hline
    Cohen-Added et al. \cite{cohen2017fast} & $S\in [n^{3/2}, n^2]$ &
    $O(n^{5/2}/S^{3/2}\lg n)$ & $\tilde{O}(S)$* \\
    \hline
    Cohen-Added et al. \cite{cohen2017fast} & $S=O(n^{5/3})$ & $O(\lg n)$ &
    $\tilde{O}(n^{5/3})$* \\
    \hline
    Gawrychowski et al. \cite{gawrychowski2017better} & $S\in [n, n^2]$ &
    $\tilde{O}(\max\{1, n^{3/2}/S\})$ & ---\\
    \hline
    Gawrychowski et al. \cite{gawrychowski2017better} & $S=O(n^{3/2})$ & $O(\lg n)$ & --- \\
    \hline
    Wulff-Nilsen \cite{wulff2010algorithms} & $S=o(n^2)$ & $O(1)$ & $o(n^2)$ \\
    \hline
  \end{tabular}
  \caption{Results for distance oracles in directed planar graphs that are not
    vertex-labeled. All results are exact. \\
    * Using Gawrychowski et al. \cite{gawrychowski2018voronoi} to speed up
  preprocessing.}
  \label{planarresults}
\end{table}

\subsection{The dynamic planar case}
TODO
