\section{Exact distance oracles}
dkosapkdpa

\section{Approximate distance oracles}
Hermelin et al. \cite{hermelin2011distance} proposed several results for approximate
distance oracles. In the first result, they give a vertex-labeled distance oracle of
expected size $O(kn^{1+1/k})$ with a stretch of $(4k-5)$ and query time $O(k)$. This can
be constructed in $O(kmn^{1/k})$ time. The data structure is an adaption of Thorup and
Zwick's distance oracle \cite{thorup2005approximate}. Given some $k$, we construct the
$k$ sets $V=A_{k}\subseteq A_{k-1} \subseteq \dots \subseteq A_0$. \\
In another result, they give a distance oracle which depend on both $n$ and $\ell$. This has
expected size $O(knl^{1/k})$ with query time $O(k)$, but with an exponential stretch of
$(2^k-1)$. This is constructed in $O(kmn^{k/(2k-1)})$. \\
In their third result, they give an oracle capable of handling label changes. This has
expected space $O(kn^{1+1/k})$, stretch $(2\cdot3^{k-1}+1)$, query time $O(k)$ and
handles label changes in $O(kn^{1/k}\lg \lg n)$. \\
TODO: Short description on approach. Detailed if we use the ideas. \\
\\
Chechik \cite{chechik2012improved} improved the latter two results. Namely, he managed to
reduce stretch from exponential down to polynomial. He showed how to construct a distance oracle
of size $O(kn\ell^{1/k})$, stretch $(4k-5)$  and query time $O(k)$, which can be
constructed in $O(m\cdot \min\{n^{k/(2k-1)}, \ell\})$. \\
He also improved a dynbamic vertex-labeled distance oracle capable of handling label changes in
$\tilde{O}(n^{1/k})$ time with stretch $(4k-5)$. He also slightly improved space to
$\tilde{O}(n^{1+1/k})$. \\
TODO: Short description on approach. Detailed if we use the ideas.
