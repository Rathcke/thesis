\section{Missing proofs}
\subsection{$VD^*$ is a tree}\label{a2}
\begin{lemma}
  $VD*$ is a tree.
\end{lemma}
\textbf{Proof}. Suppose that $VD^*$ contains a cycle $C^*$. Since the degree of each copy
of $h^*$ is one, the cycle does not contain $h^*$. Therefore, since all the sites are on
the boundary of hole $h$, the vertices of $P$ enclosed by $C^*$ are in a Voronoi cell
that contains no site, a contradiction. To prove $VD^*$ is connected, observe that in
$VD^*$, every Voronoi cell is a face (cycle) going through $h^*$. Let $C^*$ denote this
cycle. If $C^*$ is disconnected in $VD^*$ then $C^*$ must visit $h^*$ at least twice. But
this implies that the cell corresponding to $C^*$ contains more than a single site,
contradiction to our assumption. Thus, the boundary vertex of every Voronoi cell is a
connected subgraph of $VD^*$. Since the boundaries of the cell of $s_i$ and the cell of
$s_{i+1}$ both contain the dual of the edge $s_is_{i+1}$, it follows that the entire
modified $VD^*$ is connected. \qed

\subsection{Point location in $VD^*$}\label{a1}
\begin{lemma}
  Let $s$ be the site such that $v\in \text{Vor}(s)$. If $T^*$ contains all edges of
  $VD^*$ incident to $\text{Vor}(s)$, and if $v$ is closer to site $s_i$ than to the
  sites $s_{i-1}$ and $s_{i+1}$ (indices modulo $3$), then one of the following is true:
  \begin{itemize}
    \item $s=s_i$
    \item $v$ is right of $p_i$ and all boundary edges of $\text{Vor}(s)$ are contained
      in in $T_i^*$
    \item $v$ is left of $p_i$ and all boundary edges of $\text{Vor}(s)$ are contained
      in in $T_{i+1}^*$
  \end{itemize}
\end{lemma}
\textbf{Proof}. In the following, let $rev(q)$ denote the reverse path of $q$.
\indent Let $p$ be the shortest path from $s_i$ to $v$. If $p$ is a subpath of $p_i$,
then $s=s_i$. Assume that $p$ emanates right of $p_i$ (the other case is symmetric).
First observe that the path consisting of the concatenation $p_i \circ rev(p_{i-1})$
intersects $VD^*$ only at $v^*$. This is because, apart from the artificial arc $y_jv^*$,
each shortest path $p_i$ is entirely contained in the Voronoi cell of $s_i$. Therefore,
none of the subtrees $T_{i'}^*$ contains an edge dual to $p_i \circ rev(p_{i-1})$. Since
the path $p_i \circ rev(p_{i-1})$ starts on $h$, ends on $h$ and contain no other
vertices of $h$. It partitions the embedding into two subgraphs, one to the right of $p_i
\circ rev(p_{i-1})$, and the other to its left. Since $e_i^*$ is the only edge of $T^*$
that emanates right of $p_i\circ rev(p_{i-1})$, the only edges of $T^*$ in the right
subgraph are those of $T_i^*$. \\
Next observe that $p$ does not cross $p_i$ (since shortest paths from the same source do
not cross), and does not cross $p_{i-1}$ (since $v$ is closer to $s_i$ that $s_{i-1}$.
Since we assumed $p$ emanates right of $p_i$, the only edges of $T^*$ whose duals belong
to $p$ are edges of $T_i^*$. Consider the last edge $e^*$ of $p$ that is not strictly in
$\text{Vor}(s)$. If $e^*$ does not exist then $p$ consists only of edges of
$\text{Vor}(s_{i})$, so $s=s_i$. If $e^*$ does exist then it is incident to
$\text{Vor}(s)$. By the statement of the Lemma all edges of $VD^*$ incident to
$\text{Vor}(s)$ are in $T^*$. Therefore, by the discussion above, $e^*\in T_i^*$. We have
established that some edge of $VD^*$ incident to $\text{Vor}(s)$ is in $T_i^*$. It
remains to show that all such edges are in $T_i^*$. The only two Voronoi cells that are
partitioned by the path $p_i\circ rev(p_{i-1})$ are $\text{Vor}(s_i)$ and
$\text{Vor}(s_{i-1})$. Since $v$ is closer to $s_i$ than to $s_{i-1}$, $s\neq s_{i-1}$.
Hence either $s=s_i$, or all the edges of $VD^*$ incident to $\text{Vor}(s)$ are in
$T_i^*$. \qed
