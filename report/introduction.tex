\section{Introduction}\label{sec:introduction}
Efficient algorithms on networks ... Road mapping .. Graphs. \\
As these networks can be quite large, one problem is constructing a compact \textit{distance
oracle}, which is a data structure that is able to either give exact or approximate
shortest paths between two nodes quickly while not taking up too much space. They were
introduced by Thorup and Zwick \cite{thorup2005approximate}. This problem has since been
extensively studied. However, a variant of this problem
was introduced by Hermelin et al. \cite{hermelin2011distance}. In the variant, we seek a
compact distance oracle on \textit{vertex-labeled} graphs. For this scenario, each node has a label (e.g. "gas
station" or "supermarket") and the oracles answer distance queries between a node and a
label. This generalizes the normal distance oracle problem since when each node is
associated with a unique label it is the same problem. \\
