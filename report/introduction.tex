\section{Introduction}\label{introduction}
Efficient algorithms on networks are playing an increasing important role today, and many
networks have structural properties that can be exploited to allow for much faster
algorithms than those in general graphs. One such class of networks are planar graphs, which have importance within
areas such as road transportation, computer vision and many more.. \\
As these networks can be quite large, one problem is constructing a compact \textit{distance
oracle}, which is a data structure that is able to either give exact or approximate
shortest paths between two nodes quickly while not taking up too much space. They were
introduced by Thorup and Zwick \cite{thorup2005approximate}. This problem has since been
extensively studied. However, a variant of this problem
was introduced by Hermelin et al. \cite{hermelin2011distance}. In the variant, we seek a
compact distance oracle on \textit{vertex-labeled} graphs. For this scenario, each node has a label (e.g. "gas
station" or "supermarket") and the oracles answer distance queries between a vertex and
the nearest vertex with some label $\lambda$. This generalizes the normal distance oracle problem since when each node is
associated with a unique label it is the same problem.
When analysing the performance of
distance oracles, we typically look at three things. The first is \textit{query} time.
This is the amount of time it takes to return the shortest path between two given
vertices or the shortest path between a vertex and the nearest $\lambda$-labeled vertex.
The second is the \textit{space} required. This is the information stored in the data
structure we have access to at query time. The last is \textit{preprocessing}. This is
the time it takes to "handle" a graph, as to determine what information we will store
in the data structure. \\
\\
This thesis focuses on vertex-labeled distance oracles in planar graphs that can provide
exact distances between a vertex and a desired label $\lambda$. In Section \ref{exactPlanar} we
give two oracles for the static setting. The first oracle described in Section \ref{oracle1}
exploits the fact that
the number of labels $\ell$ is less than the number of vertices in the graph $n$. This
allows us to construct a more compact oracle with smaller query times. Specifically, we
can in $O(n^2)$ time construct an oracle of size $O(nl^{2/3})$ that answer queries in
$O(\ell^{1/3})$ time. The second oracle
in Section \ref{oracle2}, is a more technical oracle, which achieves better query times, but has no dependence
on $\ell$. This oracle does not work for all distributions of the label set $L$. It can
in $O(n^{3/2})$ time preprocess a planar graph into an oracle requiring $O(n^{3/2})$
space that can answer queries in $O(\text{polylog}(n))$ time. \\
In Section \ref{dynamicPlanar}, we consider the dynamic setting. In Section
\ref{oracle3}, we extend the first oracle to handle edge
weight updates in amortized $\tilde{O}(n^{2/3})$ time and still give exact shortest path in subsequent distance queries. In Section \ref{oracle4}, we extend the second oracle to support label
changes in expected amortized $O(1)$ time. That is, we can change a label $\lambda_i$ to $\lambda_j$ while maintaining
exact shortest paths.

\subsection{Notation}\label{notation}
For a graph $G=(V,E)$ with label set $L=\{\lambda_0, \lambda_1, \dots, \lambda_\ell\}$, we define $|V|=n$, $|E|=m$ and $|L|=\ell$. A
distance $\delta(v,\lambda)$ for a $v\in V$ and $\lambda\in L$ is the distance between
the node $v$ and the nearest $\lambda$-labeled node. Likewise, we use $\delta(u,v)$ to denote
the minimal distance between two nodes $u,v\in V$. \\
We use $Q(v,\lambda)$ or $Q(u,v)$ to denote the queries "What is $\delta(v,\lambda)$?" or
"What is $\delta(u,v)$?" respectively. \\
