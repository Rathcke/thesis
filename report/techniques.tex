\section{Techniques for planar graphs}\label{techniques}
In this section, we describe techniques popularly used for algorithms in planar graphs.

\subsection{Separators in planar graphs}
A \textit{Jordan curve} is a simple closed curve in the plane. Given a planar graph $G$,
a Jordan curve separator is a curve that intersects $G$ only at vertices. In 1984, Miller
\cite{miller1984finding} showed that for any planar graph with $n$ vertices, there is a Jordan curve
separator of size $O(\sqrt{n})$ such that each new component contains at most $2n/3$
vertices. He further showed that such a separator can be found in $O(n)$ time. \\
Frederickson introduced the \textit{$r$-division} for planar
graphs. Given an $r\in (0,n)$, the division is a decomposition of the graph into
$O(n/r)$ edge-induced subgraphs (\textit{pieces}), where each piece contains $O(r)$ vertices and $O(\sqrt{r})$ \textit{boundary vertices}. A
boundary vertex is a vertex that belongs to multiple pieces. Frederickson showed that
such a division exists for any planar graphs and can be found in $O(n\lg n)$ time. This
was achieved by applying the separator theorem of Lipton and Tarjan \cite{lipton1979separator}. This was further developed
by Klein et al. \cite{klein2013structured}, who gave a $O(n)$ time algorithm to compute
an $r$-division of a planar graph $G$ with the additional property that each piece has a constant number of \textit{holes}, which are faces of a piece that are not faces of $G$.

\subsection{Klein's multiple source shortest path}
Klein's multiple source shortest path \cite{klein2005multiple} is useful tool for planar
graphs. Given a planar graph with nonnegative weight, we can construct a data structure
of size $O(n \lg n)$ that can answer queries of the following form in $O(\lg n)$ time:
For a source node $s$ and a node $t$ on the boundary of the infinite face, find the
shortest path from $s$ to $t$. This data structure can be constructed in $O(n \lg n)$
time. TODO: More?

\subsection{Monge property}


\subsection{FR-Djikstra}
FR-Djikstra \cite{fakcharoenphol2006planar}\\
A \textit{Dense distance graph} or DDG is defined over a decomposition of a planar
graph.

\subsection{Shortest path trees}
TODO

